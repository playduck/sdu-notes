\documentclass[
    fontsize      = 11pt,
    paper         = a4,
    twoside       = false,
    parskip       = half,
    pagesize      = false,
]{scrartcl}

\author{Robin Prillwitz}

\usepackage[ngerman]{babel}
\usepackage[iso,german]{isodate}
\date{11. August 2022}

\usepackage{hyphenat}
\hyphenation{Mathe-matik wieder-gewinnen}
\usepackage[babel=true]{csquotes}
\usepackage[protrusion=true,expansion,tracking=true,nopatch=eqnum]{microtype}

\usepackage{amsmath}
\usepackage{amssymb}
\usepackage[locale=DE]{siunitx}

\usepackage[outputdir=/Users/robin/Documents/sdu-notes/temp/sdu]{minted}

\usepackage{graphicx}
\usepackage{grffile}
\graphicspath{{./img/}}

\usepackage{longtable}
\usepackage{booktabs}

\usepackage{ulem}

% % Scale images if necessary, so that they will not overflow the page
% % margins by default, and it is still possible to overwrite the defaults
% % using explicit options in \includegraphics[width, height, ...]{}
% \setkeys{Gin}{width=\maxwidth,height=\maxheight,keepaspectratio}

\usepackage[usenames,dvipsnames,svgnames,table]{xcolor}

\usepackage{tikz}
\usetikzlibrary{patterns}
\usetikzlibrary{calc}
\usetikzlibrary{shapes}
\usetikzlibrary{decorations.markings}
\usetikzlibrary{arrows,automata,backgrounds,petri}
\usepackage[european, betterproportions]{circuitikz}
\usepackage{pgfplots}
\usepackage{pgfplotstable}
\usepackage{pgfgantt}
\usepackage{pgfornament}
\pgfplotsset{
  compat=1.18, % lastest release as of 2022-07-02
}

\usepackage{scrlayer}
\usepackage[]{scrlayer-scrpage}
\ohead{11. August 2022}
\chead{\lowercase{\scshape{Sdu}}}
\ihead{Robin Prillwitz}
\ofoot*{\pagemark}
\cfoot*{}

\usepackage[utf8]{inputenc}
\usepackage[T1]{fontenc}
\usepackage{fontspec}
\usepackage{textcomp}

\setsansfont[
    Scale       = MatchLowercase,
    ScaleAgain  = 1.08,
    Ligatures   = TeX
]{Helvetica Neue}

\setmonofont[
    Scale       = MatchLowercase,
    Ligatures   = TeX,
    Contextuals = {Alternate}
]{Fira Code}

\setmainfont[
    Scale       = MatchLowercase,
    UprightFont =  *-Regular,
    BoldFont    =  *-Bold,
    ItalicFont  =  *-It,
    Ligatures   = TeX
]{Minion Pro}

\providecommand{\tightlist}{%
  \setlength{\itemsep}{0pt}\setlength{\parskip}{0pt}}

\usepackage[hidelinks]{hyperref}

\begin{document}

\hypertarget{artificial-learning}{%
\section{Artificial learning}\label{artificial-learning}}

\hypertarget{plasticity}{%
\subsection{Plasticity}\label{plasticity}}

\textbf{Neuroplasticity}: Ability for the brain to re-organize itself in
both \emph{structure} and \emph{function} over time due to external and
internal events. \textbf{Neuroplasticity} is~mechanism behind
``\emph{learning}'' and is happening continuously.

\begin{longtable}[]{@{}ll@{}}
\toprule
\textbf{Structural Plasticity} & ~\textbf{Functional
Plasticity} \\ \addlinespace
\midrule
\endhead
new neural connections & changing existing connections \\ \addlinespace
long-term changes & short term changes \\ \addlinespace
\bottomrule
\end{longtable}

\textbf{Plasticity} happens on all levels from cortical down to the
synaptic level.

\begin{itemize}
\tightlist
\item
  \textbf{cortical}: changing stimulus from limbs triggers different
  existing neurons
\item
  \textbf{synaptic}: changing amount of gates on post-synaptic neurons'
  dendrites
\end{itemize}

\hypertarget{synaptic-strength-in-functional-plasticity}{%
\subsection{\texorpdfstring{\textbf{synaptic strength} in functional
plasticity}{synaptic strength in functional plasticity}}\label{synaptic-strength-in-functional-plasticity}}

\hypertarget{long-term-potentiation-ltp}{%
\subsubsection{\texorpdfstring{Long Term Potentiation
(\emph{LTP})}{Long Term Potentiation (LTP)}}\label{long-term-potentiation-ltp}}

\textbf{HFS}: \(100\) Pulses (over \(1\si{s} \rightarrow 100\si{Hz}\))
as an input to a neuron. The neuron is resting at \(t=0\). The
\textbf{HFS} hits the neuron resulting in an instantaneous output, the
\textbf{LTP}. The neurons output jumps, then receedes and continues to
saturate (Only as long as the \textbf{HFS} is continous.) The
\textbf{synaptic strength} is the chance the output is increased.

A lot of fast input \(\rightarrow\) Big changes and high learning

\textbf{LTP \emph{increases} synaptic strength}

\hypertarget{long-term-depressino-ltd}{%
\subsubsection{\texorpdfstring{Long Term Depressino
(\emph{LTD})}{Long Term Depressino (LTD)}}\label{long-term-depressino-ltd}}

The Inverse, to decrease the \textbf{synaptic strength} an \textbf{LFS}
(\(900\) Pulses \(15\si{min} \rightarrow 1\si{Hz}\)) is sent. The neuron
responsds, dips and saturates in a depression.

Low data \(\rightarrow\) Low learning

\textbf{LTD \emph{decreases} synaptic stregth}

\hypertarget{chemical-basis}{%
\subsubsection{Chemical basis}\label{chemical-basis}}

\textbf{LTP} and \textbf{LTD} result in synapeses by creating or
destroying gates at the pos-synaptic terminal respectively.

\hypertarget{hebbian-learning-model}{%
\subsection{Hebbian learning model}\label{hebbian-learning-model}}

Efficiency describes the likelyhood if a presynaptic neuron spiking and
exciting it's postsynaptic neuron. The likelyhood of the post-synaptic
neuron firing after having been exicted is increased. More firing
together \(\rightarrow\) more likely to fire together in the future.
They spiking is, however, \emph{not necessarily causal}. At high
efficiency the spiking of both neurons are \textbf{temporally
correlated}. The spiking is \textbf{associative} and
\textbf{unsupervised}.

\textbf{Neurons that fire together, wire together.}

\hypertarget{simple-mathematical-model}{%
\subsubsection{Simple mathematical
model}\label{simple-mathematical-model}}

\[\frac{\mathit{d}\omega_1}{\mathit{d}t} = \mu \cdot v \cdot u_1\]

\begin{itemize}
\tightlist
\item
  \(\omega\): dsecribes the synaptic strength / weight
\item
  \(\frac{\mathit{d}\omega_1}{\mathit{d}t}\): (not a derivative), Change
  in synaptic weight
\item
  \(\mu\): Learnig rate (\(\mu \ll 1\) to avoid ``exploding learning
  problem'')
\item
  \(v\): Output of post-synaptic neuron
\item
  \(u_1\): Output of pre-synaptic neuron / input to post-synaptic neuron
\end{itemize}

\[\omega_n = \omega_{n-1} + \frac{\mathit{d}\omega_{n-1}}{\mathit{d}t} = \omega_{n-1} + \mu \cdot v \cdot u_{n-1} \]

\textbf{Problem}: \(\omega_1\) is always increasing, unstable \sout{but
biologically correct}. This is an open control loop.

As this is unsupervised we don't have an error term and can't simply
stop when the model is ``good enough''.

\hypertarget{ltp}{%
\subsubsection{LTP}\label{ltp}}

The further the amount of time between two spikes firing the more the
weight changes. A high \(\delta t\) results in little change, a small
\(\delta t\) results in large changes. At \(\delta t = 0\) maximal
change occurs. The simple model only results in positive change, thus
unstable.

\hypertarget{input-correlation-learning-ico}{%
\subsection{\texorpdfstring{Input correlation learning
(\emph{ICO})}{Input correlation learning (ICO)}}\label{input-correlation-learning-ico}}

Learning rule
\[\frac{\delta w_a}{\delta t} = \eta \cdot f \left( A, t\right) \otimes \frac{\delta f \left( B, t\right) }{\delta t}\]

\begin{itemize}
\tightlist
\item
  \(\eta\): learning rate
\item
  \(f\left( A, t\right) \otimes \frac{\delta f \left( B, t\right) }{\delta t}\):
  Temporal correlation
\item
  \(otimes\): cross correlation
\item
  \(A\): Predictive signal
\item
  \(B\): Reflex signal
\item
  \(Y\): Neuron Ouput
\item
  \(w_a\) weight between \(A\) and \(Y\)
\item
  \(f\) output function of a neuron (including the sigmoid)
\end{itemize}

If we'd like to stop the learning we can assume \(B\) to be constant. We
cannot guarantee \(B \rightarrow 0\) (to stop learning) but we can take
the derivative to stop learning once stimulus ceases change.

This Algorithm \textbf{will converge} to the correct weight.

Output signal is the weighted sum.
\[Y(t) = w_a \cdot f \left( A, t\right) + f \left( B, t\right)\]

\hypertarget{perceptron-learning}{%
\subsubsection{Perceptron learning}\label{perceptron-learning}}

Learning by updating input weights only. Update done using
\textbf{gradient descent}.

Update weight in proportion to contribution to the output. Contribution
is the change in error \(E\) für a given change in \(w\), where the mean
squared error is defined as \[E = \frac{1}{2} \left( t-v\right)^2\]

\begin{itemize}
\tightlist
\item
  \(t\): target output
\item
  \(v\): actual output
\end{itemize}

Determining error requires a known correct output.

\(\rightarrow\) \textbf{supervised learning}

\begin{center}\rule{0.5\linewidth}{0.5pt}\end{center}

Derivative of error is a gradient of \(E\). Finding the global minimum
is done using gradient descent. The error gradient for a sigmoid is
given by \begin{align*}
\frac{\mathit{d}E}{\mathit{d}w} &= \frac{1}{2}\cdot 2 \frac{\mathit{d}(t-v)}{\mathit{d}w}(t-v) \\[2ex]
&= v (1-v)(t-v)
\end{align*}

Updates on weights are done using
\[ \frac{\mathit{d}w_i}{\mathit{d}t} = \mu \cdot v(1-v) (t-v) \cdot u_i\]

\end{document}
