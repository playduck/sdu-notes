\documentclass[
    fontsize      = 11pt,
    paper         = a4,
    twoside       = false,
    parskip       = half,
    pagesize      = false,
]{scrartcl}

\author{Robin Prillwitz}

\usepackage[ngerman]{babel}
\usepackage[iso,german]{isodate}
\date{10. August 2022}

\usepackage{hyphenat}
\hyphenation{Mathe-matik wieder-gewinnen}
\usepackage[babel=true]{csquotes}
\usepackage[protrusion=true,expansion,tracking=true,nopatch=eqnum]{microtype}

\usepackage{amsmath}
\usepackage{amssymb}
\usepackage[locale=DE]{siunitx}

\usepackage[outputdir=/Users/robin/Documents/sdu-notes/temp/sdu]{minted}

\usepackage{graphicx}
\usepackage{grffile}
\graphicspath{{./img/}}

% % Scale images if necessary, so that they will not overflow the page
% % margins by default, and it is still possible to overwrite the defaults
% % using explicit options in \includegraphics[width, height, ...]{}
% \setkeys{Gin}{width=\maxwidth,height=\maxheight,keepaspectratio}

\usepackage[usenames,dvipsnames,svgnames,table]{xcolor}

\usepackage{tikz}
\usetikzlibrary{patterns}
\usetikzlibrary{calc}
\usetikzlibrary{shapes}
\usetikzlibrary{decorations.markings}
\usetikzlibrary{arrows,automata,backgrounds,petri}
\usepackage[european, betterproportions]{circuitikz}
\usepackage{pgfplots}
\usepackage{pgfplotstable}
\usepackage{pgfgantt}
\usepackage{pgfornament}
\pgfplotsset{
  compat=1.18, % lastest release as of 2022-07-02
}

\usepackage{scrlayer}
\usepackage[]{scrlayer-scrpage}
\ohead{10. August 2022}
\chead{\lowercase{\scshape{Sdu}}}
\ihead{Robin Prillwitz}
\ofoot*{\pagemark}
\cfoot*{}

\usepackage[utf8]{inputenc}
\usepackage[T1]{fontenc}
\usepackage{fontspec}
\usepackage{textcomp}

\setsansfont[
    Scale       = MatchLowercase,
    ScaleAgain  = 1.08,
    Ligatures   = TeX
]{Helvetica Neue}

\setmonofont[
    Scale       = MatchLowercase,
    Ligatures   = TeX,
    Contextuals = {Alternate}
]{Fira Code}

\setmainfont[
    Scale       = MatchLowercase,
    UprightFont =  *-Regular,
    BoldFont    =  *-Bold,
    ItalicFont  =  *-It,
    Ligatures   = TeX
]{Minion Pro}

\providecommand{\tightlist}{%
  \setlength{\itemsep}{0pt}\setlength{\parskip}{0pt}}

\usepackage[hidelinks]{hyperref}

\begin{document}

\hypertarget{artificial-neural-brains}{%
\section{Artificial neural brains}\label{artificial-neural-brains}}

\hypertarget{braitenberg-vehicles}{%
\subsection{Braitenberg Vehicles}\label{braitenberg-vehicles}}

\begin{itemize}
\tightlist
\item
  \textbf{Ipsilateral}: Connections on same side
\item
  \textbf{Contralateral}: Connections cross sides
\item
  \textbf{Excitatory}: Input Increases \textrightarrow~Output Increases
\item
  \textbf{Inhibitory}: Input Increases \textrightarrow~Output Decreases
\end{itemize}

Vehicle emulates simple \(P\)-type control.

Mathematical model includes:

\begin{itemize}
\tightlist
\item
  \(s_x\): Sensor value
\item
  \(v_x\): Output value
\item
  \(k\): Linear proportional gain
\end{itemize}

Mathematical example implementations:

\begin{itemize}
\tightlist
\item
  \textbf{Ipsilateral}: \(v_{\text{left}} \propto s_{\text{left}}\)
\item
  \textbf{Contralateral}: \(v_{\text{left}} \propto s_{\text{right}}\)
\item
  \textbf{Excitatory}: \(v \propto s\)
\item
  \textbf{Inhibitory}: \(v \propto \frac{1}{s}\)
\end{itemize}

\textbf{Pathplanning}: Finding path from known start to known end
including known obstacles.

Complex behavior emerges by combining multiple weighted control loops
running in parallel.

\end{document}
