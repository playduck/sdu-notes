\documentclass[
    fontsize      = 11pt,
    paper         = a4,
    twoside       = false,
    parskip       = half,
    pagesize      = false,
]{scrartcl}

\author{Robin Prillwitz}

\usepackage[ngerman]{babel}
\usepackage[iso,german]{isodate}
\date{13. August 2022}

\usepackage{hyphenat}
\hyphenation{Mathe-matik wieder-gewinnen}
\usepackage[babel=true]{csquotes}
\usepackage[protrusion=true,expansion,tracking=true,nopatch=eqnum]{microtype}

\usepackage{amsmath}
\usepackage{amssymb}
\usepackage[locale=DE]{siunitx}

\usepackage[outputdir=/Users/robin/Documents/sdu-notes/temp/sdu]{minted}

\usepackage{graphicx}
\usepackage{grffile}
\graphicspath{{./img/}}

\usepackage{longtable}
\usepackage{booktabs}

\usepackage[normalem]{ulem}

% % Scale images if necessary, so that they will not overflow the page
% % margins by default, and it is still possible to overwrite the defaults
% % using explicit options in \includegraphics[width, height, ...]{}
% \setkeys{Gin}{width=\maxwidth,height=\maxheight,keepaspectratio}

\usepackage[usenames,dvipsnames,svgnames,table]{xcolor}

\usepackage{tikz}
\usetikzlibrary{patterns}
\usetikzlibrary{calc}
\usetikzlibrary{shapes}
\usetikzlibrary{decorations.markings}
\usetikzlibrary{arrows,automata,backgrounds,petri}
\usepackage[european, betterproportions]{circuitikz}
\usepackage{pgfplots}
\usepackage{pgfplotstable}
\usepackage{pgfgantt}
\usepackage{pgfornament}
\pgfplotsset{
  compat=1.18, % lastest release as of 2022-07-02
}

\usepackage{scrlayer}
\usepackage[]{scrlayer-scrpage}
\ohead{13. August 2022}
\chead{\lowercase{\scshape{Sdu}}}
\ihead{Robin Prillwitz}
\ofoot*{\pagemark}
\cfoot*{}

\usepackage[utf8]{inputenc}
\usepackage[T1]{fontenc}
\usepackage{fontspec}
\usepackage{textcomp}

\setsansfont[
    Scale       = MatchLowercase,
    ScaleAgain  = 1.08,
    Ligatures   = TeX
]{Helvetica Neue}

\setmonofont[
    Scale       = MatchLowercase,
    Ligatures   = TeX,
    Contextuals = {Alternate}
]{Fira Code}

\setmainfont[
    Scale       = MatchLowercase,
    UprightFont =  *-Regular,
    BoldFont    =  *-Bold,
    ItalicFont  =  *-It,
    Ligatures   = TeX
]{Minion Pro}

\providecommand{\tightlist}{%
  \setlength{\itemsep}{0pt}\setlength{\parskip}{0pt}}

\usepackage[hidelinks]{hyperref}

\begin{document}

\hypertarget{questions-answers}{%
\section{Questions \& Answers}\label{questions-answers}}

\hypertarget{biologically-inspired-robotics}{%
\subsection{Biologically inspired
robotics}\label{biologically-inspired-robotics}}

\begin{itemize}
\item
  \emph{What is biologically-inspired robotics?}

  Using biological systems as inspiration for robotic design.
\item
  \emph{What is biorobotics?}

  Using the biologically inspired robotic system to better understand
  the original biological system.
\item
  \emph{What is the difference between biorobotics and
  biologically-inspired robotics?}

  Biorobotics combines existing biological systems with mechanical
  systems. Biologically-inspired robotics takes inspiration of
  biological systems for robotics without combining the two.
  \textbf{(?)}
\item
  \emph{What is the reality check in biorobotics?}

  Complex biological behaviour doesn't arise from complex systems but
  rather simple, yet dedicated, systems.
\item
  \emph{What is neurorobotics?}

  The study and application of science and technology of embodied,
  autonomous, brain-inspired algorithms.
\item
  \emph{What is a neurorobot? Can you give an example of a neurorobot?}

  \textbf{(?)}
\item
  \emph{What is embodied AI?}

  A purpose built mechanical system in conjunction with an AI
  controller.
\item
  \emph{Can you explain the three-layer embodied AI architecture?}

  \begin{itemize}
  \tightlist
  \item
    The controller acting as a supervisor: The brain
  \item
    The mechanical input and outputs: Motors and Sensor
  \item
    The environment
  \end{itemize}
\item
  \emph{Can you give an example of embodied AI in humans/animals? Can
  you give an example of embodied AI in robots?}

  \textbf{(?)}
\item
  \emph{Why is the environment important in embodied AI?}

  An enviornment influences an agent. An agent must overcome external
  influences or take advantage of them. Example: Seagulls in hovering in
  gusts of wind without effort.
\item
  \emph{What are model-based and model-free approaches in biorobotics?
  Can you use both together in one robot? Can you give an example?}

  \begin{itemize}
  \tightlist
  \item
    \textbf{Model based}: A mathematical model defines the entire
    robotic system.
  \item
    \textbf{Model free}: The robot acts according to some defined rules
    but ``figures out'' how to achive it's objective by itself.
  \end{itemize}

  Both appoaches can be used together. A model-based approach defining
  the systems baseline behviour with the model-free system adapting to
  external changes.

  Examples: - \textbf{Model based}: PID Controller - \textbf{Model
  free}: \textbf{(?)} - \textbf{Both}: \textbf{(?)}
\end{itemize}

\hypertarget{neurons}{%
\subsection{Neurons}\label{neurons}}

\begin{itemize}
\item
  \emph{What is a neuron?}

  A biological braincell responsible for processing.
\item
  \emph{What are the different types of neurons?}

  \begin{itemize}
  \tightlist
  \item
    \textbf{Unipolar}: Inputs, from outside the brain (found in insects)
  \item
    \textbf{Bipolar}: Inputs, from senses (eyes, ears)
  \item
    \textbf{Multipolar}: Within the brain and as outputs to muscles

    \begin{itemize}
    \tightlist
    \item
      \textbf{Pyramidal}: complex though (Cerebrum)
    \item
      \textbf{Purkinje}: reactive action (Cerebellum)
    \end{itemize}
  \end{itemize}
\item
  \emph{How does a neuron work/transmit a signal through itself?}

  Using electrical spikes, presumeably through the spike signal's
  frequency and timing charachteristics, not by it's shape, amplitude or
  phase. \textbf{(?)}
\item
  \emph{How does a neuron forward a signal to other neuron(s)?}

  By using chemical reactions at synapses.
\item
  \emph{What is a synapse?}

  A non-physical connection between two neurons.
\item
  \emph{How does a synapse transmit a signal?}

  The electric signal in the pre-synaptic neuron releases
  neurotransmitters into the synaptic chasm. These travel through the
  brain fluid to the post-synaptic neurons neuroreceptors. If enough
  receptors get stimulated they create an electric spike.
\item
  \emph{What is an action potential?}

  A group of various electric signals collected by a neuron.
  \textbf{(?)}
\item
  \emph{Describe how an action potential is generated.}

  By the neuroreceptors in the post-synaptic neuron.
\item
  \emph{Describe how the different phases of the action potential are
  generated.}

  \begin{itemize}
  \tightlist
  \item
    The \textbf{resting phase} is the default state.
  \item
    The \textbf{depolarization phase} occurs once the \(Na+\) gates
    reach a voltage-threshold. A voltage overshoot occurs as reactions
    are not instantenous.
  \item
    The \textbf{repolarization phase} follows showing a decline in
    voltage as the Sodium gates saturate and repell ions, thus allowing
    Potassium (\(K+\)) to enter.
  \item
    The \textbf{hyperpolarization phase} returns to the resting state as
    the voltage reaches an equilibrium.
  \end{itemize}
\item
  \emph{What is resting
  potential/depolarisation/repolarisation/hyperpolarisation? What
  mechanism(s) causes each phase to be generated?}

  \textbf{See above answer}.
\item
  \emph{Why is the resting potential negative?}

  As ions leak through the neurons membrane the neuron takes on a more
  negative charge in respect to the outside fluid.
\item
  \emph{How is information encoded in spikes?}

  Presumeably in its frequency and timing charachteristics.
\item
  \emph{What is a perceptron?}

  A quantized model of a neurons behavior. Consisting of multiple
  weighted inputs getting summed and passed through an activation
  function.
\item
  \emph{What is an activation?}

  The activation describes the weighted sum of the perceptrons inputs.
  \(z = \sum_{i=1}^n \omega_i \cdot u_i\)
\item
  \emph{What is an activation function? Why do we need it/What will
  happen if I don't use it?}

  A mathematical function limiting the perceptrons activation (weighted
  sum) to a knwon output space.
\item
  \emph{What are the different types of activation functions and their
  advantages/drawbacks? Which one will you choose and why?}

  \begin{align*}
    \text{RelU } &= \begin{cases} 0 \text{ for } x < 0\\ x \text{ for } x \geq 0\end{cases}\\[2ex]
    \text{Heaviside } &= \begin{cases} 0 \text{ for } x < 0\\ 0.5 \text{ for } x = 0\\ 1 \text{ for } x > 0\end{cases}\\[2ex]
    \text{Linear } &= x\\
    \text{Sigmoid} &= \frac{1}{1+e^{-x}}
    \end{align*}

  \begin{itemize}
  \tightlist
  \item
    Heaviside limits the output to a binary state, severly reducing
    granularity.
  \item
    Linear and RelU are computationally efficient but may grow
    unbounded.
  \item
    Sigmoids limit the effective range and don't quantize dramatically.
  \end{itemize}
\item
  \emph{Why do we use a sigmoid activation function (think about it from
  both computer implementation and biology perspectives)?}

  It binds the output to a manageable range which avoids overflow and
  offers great precicion using IEE754. It also models neuron's own
  saturation.
\item
  \emph{What is the basic difference between biological neural
  processing and artificial neural processing?}

  Biological neural processing can operate many billions (or more)
  neurons in parallel. Computing with artifical neurons cannot currently
  achieve this.
\item
  \emph{What is a Hodgekin-Huxley model? What
  similarities/dissimilarities does it have with a biological neuron?
  What drawbacks does it have in terms of implementation?}

  An electronic circuit aiming to reproduce neurons' voltage spikes. It
  models different Ion-charges and resistances using batteries and
  resistors acting upon a cpacitor and receiving an external input.

  The model only offers one input. It breaks down with too-high currents
  failing to simulate accordingly.

  It's implementation is computationally incredibly expensive requring
  mutliple equations to be evaluated for every timestep of the voltage
  signal.
\end{itemize}

\hypertarget{braitenberg-vehicles}{%
\subsection{Braitenberg vehicles}\label{braitenberg-vehicles}}

\begin{itemize}
\tightlist
\item
  \emph{What is a Braitenberg vehicle? Can you give an example of one
  and how it behaves?}
\end{itemize}

A simple concept of a neurorobot consisting of two powered wheels
receiving input from two sensors placed at the top left and right of the
vehicle. The connections between the sensors and wheels dictate the
vehicles emerging behaviour.

An agressive vehicle features \textbf{contralateral} and
\textbf{excitatory} connetions and agressively manouvers towards the
sensor's gratests stimulus. The \glqq love\grqq vehicle consists of
\textbf{ipsilateral} and \textbf{inhibitory} connections resulting in it
seeking the source of gratests stimulus but slowly coming to a halt when
approaching it.

\begin{itemize}
\tightlist
\item
  \emph{Are there any advantages of using Braitenberg vehicles?}
\end{itemize}

Their rather complex behavior emerging from very simplistic rules make
them computationally trivial.

They enable emulating behaviors of simple insects.

\hypertarget{learning}{%
\subsection{Learning}\label{learning}}

\begin{itemize}
\item
  \emph{What is neuroplasticity? What types of neuroplasticity are
  there?}
\item
  \emph{What is structural/functional plasticity? What are the
  differences between the two?}
\item
  \emph{Can you give an example of structural/functional plasticity?}
\item
  \emph{What is Long Term Potentiation? What is Long Term Depression?
  Can you say something about the chemical process and any changes
  underlying LTP and LDP?}
\item
  \emph{What is Hebbian learning? What are its drawbacks as a model for
  learning?}
\item
  \emph{What is Spike-Timing Dependent Plasticity?}
\item
  \emph{Do you know other forms of functional plasticity?}
\item
  \emph{What type of plasticity is Hebbian learning (homo-, hetero-,
  non- }or homeostatic)?*
\item
  \emph{What is ICO learning? What type of synaptic plasticity is it and
  why? Can you give an example of how ICO learning can be used in
  robots?}
\item
  \emph{How do perceptrons learn? What is the fundamental difference
  between ICO learning and perceptron learning (think in terms of how
  gradient descent works vs how the ICO learning rule works)?}
\item
  \emph{What is a Multi-Layer Perceptron (MLP)?}
\item
  \emph{What is the backpropagation algorithm and how does it work? Why
  is it called a gradient descent method?}
\end{itemize}

\end{document}
